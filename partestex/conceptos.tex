\chapter{Antecedentes}%
\label{cha:Antecedentes}
En éste capítulo se describirán los conceptos base para la sección del marco teórico

\section{Conceptos base}

En ésta sección se definirán los conceptos de sistema, simulación y ciclo, las bases de los sistemas energéticos que serán modelados.

\subsection{Sistema}

El concepto utilizado por \textcite{ogata1987dinamica} para definir un sistema es "la combinación de un conjunto de componentes que actúan de forma simultánea para llegar a un objetivo establecido.", a su vez, \textcite{ccengel2006termodinamica} manejan el concepto de sistema como "una cantidad de materia o una región en el espacio elegida para análisis", a su vez, la real academia de la lengua española, define un sistema como "Conjunto de cosas que relacionadas entre sí ordenadamente contribuyen a determinado objeto."

\subsubsection{Sistemas abiertos}

Los tipos de sistemas a usar, serán abordados desde la perspectiva de la termodinámica,
\textcite{ccengel2006termodinamica} define dos tipos de sistemas, los cerrados y los abiertos, los sitemas abiertos se definen como " una región elegida apropiadamente en el espacio "
e involucra un flujo másico, siendo turbinas, compensadores, ductos, ejemplos de un sistema abierto. 

\subsubsection{Sistemas cerrados}

Un sistema cerrado, también llamados masa de control, consta de una cantidad de masa fija y que ninguna otra puede entrar en dicho espacio, también conocido como frontera, ejemplos de sistemas cerrados son sistemas de cilindro-émbolo. \textcite{ccengel2006termodinamica}



\subsection{Ciclo}

\subsection{Simulación}

\subsubsection{Historia de la simulación}

\subsubsection{Tipos de simulación}

\subsubsection{Simulación por modelado matemático}


\section{Conceptos termodinámicos}

\subsection{Propiedades termodinámicas}

\subsubsection{Propiedades extensivas}

\subsubsection{Propiedades intensivas}

\subsection{Relaciones termodinámicas}

\subsection{Identidades de Maxwell}


\section{Conceptos de transferencia de calor}

\subsection{Coeficientes adimensionales}

\subsection{Conducción}

\subsection{Convección}

\subsubsection{Convección natural}

\subsubsection{Convección forzada}

\subsection{Radiación}

\subsubsection{Ley de Stefan-Boltzmann}


