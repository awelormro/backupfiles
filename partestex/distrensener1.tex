\begin{figure}[H]
    \centering
    \footnotesize
    \begin{tikzpicture}   
        % \pie[hide number, rotate=270, text=legend, color={blue!30, red!40, orange!70}]  
        \pie[hide number, rotate=270, text=legend]  
        \pie[]
        {39.2/Ciclo Combinado, 2.6/Cogeneración eficiente, 1.8/Nucleoeléctrica, 0.5/Bioenergía,7.99/Fotovoltaica,8.6/Eoloeléctrica,1.1/Geotermoeléctrica,6.1/Carboeléctrica,0.8/Combustión interna, 13.2/Térmica Convencional,14.1/Hidráulica convencional,4/Turbogás,0.01/Combustión interna}  
    \end{tikzpicture}  
    \caption{Distribución de la energía por forma de producción. \\ Tomado de SENER y CENACE (\citeyear{prodesen}).}
    \label{fig:piechart1}
\end{figure}

\pagebreak

\def\trunc#1.#2\relax{#1}
\newcommand{\slice}[6]{
        \pgfmathparse{0.5*#2+0.5*#3}
        \let\midangle\pgfmathresult

        % slice
        \draw[thick,fill=#1] (0,0) -- (#2:1) arc (#2:#3:1) -- cycle;

        % outer label
        \node[label=\midangle:#5] at (\midangle:1#6) {};

        % inner label
        \pgfmathparse{min((#3-#2-10)/110*(-0.3),0)}
        \let\temp\pgfmathresult
        \pgfmathparse{max(\temp,-0.5) + 0.8}
        \let\innerpos\pgfmathresult
        \node at (\midangle:\innerpos) {#4};
    }
    \begin{figure} [!htbp]
    \begin{center}
        \begin{tikzpicture}[scale=3]

        \newcounter{c}
        \newcounter{d}
        \foreach \p/\d/\cc in {
          13.52//cyan!90, 
          1.19/-.04/green!70, 
          3.58/.03/blue!40, % PCT/OUTER LABEL DECIMAL SHIFT/COLOR
          81.7//magenta!90}
        {
            \setcounter{a}{\theb}
            \pgfmathparse{int(100*\p)}
            \addtocounter{b}{\pgfmathresult}
            \slice{\cc}{\thea/10000*360}
            {\theb/10000*360}
            {\ifnum\expandafter\trunc\p\relax>3\relax\p\%\fi}% <- CUTOFF OF 3%
            {\ifnum\expandafter\trunc\p\relax>3\relax\else\p\%\fi}
            {\d}
        }

        \end{tikzpicture}
    \end{center}

    \caption{My pie}
\end{figure}

% \begin{center}
% \begin{tikzpicture}

% \begin{scope}
%   \draw (0,2)--(0,-2);
%   \draw (-2,0)--(2,0);
%   \newdimen\R
%   \R=1.5cm
%      \draw (36:\R)
%      \foreach \x in {36,108,180,252,324} {  -- (\x:\R) }
%    -- cycle (36:\R) node[right] {4}
%   -- cycle (108:\R) node[above] {5}
%    -- cycle (180:\R) node[below] {1}
%   -- cycle  (252:\R) node[below] {2}           -- cycle  (324:\R) node[right] {3};
% \end{scope}

% \begin{scope}[xshift=6cm]
%   \draw (0,2)--(0,-2);
%   \draw (-2,0)--(2,0);
%   \newdimen\R
%   \R=1.5cm
%      \draw (36:\R)
%      \foreach \x in {36,108,180,252,324} {  -- (\x:\R) }
%    -- cycle (36:\R) node[right] {3}
%   -- cycle (108:\R) node[above] {4}
%    -- cycle (180:\R) node[below] {5}
%   -- cycle  (252:\R) node[below] {1}           -- cycle  (324:\R) node[right] {2};
% \end{scope}

% \draw[Stealth](2.3,0) --(1,0); 
% % \draw[red,<->] (2.3,0) -- (3.8,0);

% \end{tikzpicture}
% \end{center}
% \end{document}


\begin{figure}[H]
    \centering
\begin{tikzpicture}   
  
\pie[square, rotate=270, text=inside, color={blue!40, red!30, orange!50}]  
 {46/Rice,  
     32/Wheat, 22/Other}  
   
\end{tikzpicture}

    \caption{Título genérico promedio para el humano consciente lol}
    \label{fig:cosagenérica}
\end{figure}
