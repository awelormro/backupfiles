

A su vez, \textcite{wang1} realizaron un estudio termoeconómico para un ciclo Rankine de pequeña escala, usando como parámetro las simulaciones anuales del rendimiento de la planta, variando cuatro componentes del ciclo, que fueron los colectores (colector de plato plano, tubo evacuado, y plano de tubos evacuados), las configuraciones (básico con un sistema de recuperación de calor, y ciclos subcrítico/transcrítico) los tipos de expansor (rotatorio, de tornillo, y con pistón) y un conjunto de fluidos a utilizar, que fueron refrigerantes R1234yf, R1234ze y R152a, isobutano, isopentano y butano. El sistema es de 1.1 MW de potencia instalada y buscando una eficiencia térmica del sistema de 5.5\% como objetivo, además de un obtener un costo nivelado de energía de 1 USD aproximado. Se corrieron dos simulaciones por cada variante instalada con cada uno de los sistemas en un software creado por ellos, en total utilizaron 15 estado termodinámicos con las variables mencionadas previamente, la base de los cálculos fue mediante balances de entrada y salida acondicionados por las correlaciones de transferencia de calor acorde a cada estado, probando técnicas de mejora de la eficiencia como la elevación de la temperatura al punto supercrítico, y flujos másicos variables. Los resultados obtenidos fueron:


\begin{markdown}
* Sin un sistema de almacenamiento, los sistemas pueden operar durante un rango de 5 a 6 horas diarias.
* El rango de temperatura de operación óptimo para todos los fue de 40 a 180 °C para todos los fluidos, ya que todos los líquidos previamente mencionados trabajan adecuadamente a dichas temperaturas, aunque su rango de temperatura óptima es de 140 a 230 °C, con lo que la energía dispuesta por el sistema seguía siendo menor para obtener el mejor aprovechamiento de sus cualidades termodinámicas.
* El sistema de almacenamiento trabaja de manera constante en sin variaciones de potencia con un flujo másico de 0.08 kg/s, siendo el flujo másico mejor aprovechado para el ritmo de todos los refrigerantes.
* El isobutano, butano e isopentano, tuvieron un rango de operación más óptimo gracias a sus bajas temperaturas de evaporación contra los refrigerantes que necesitaban una temperatura mayor para obtener una mejor eficiencia de operación, siendo la configuración de isobutano con calentadores de plato plano con un generador de pistón el que mejor producción tuvo, con 955 kWh de producción anual.
\end{markdown}
