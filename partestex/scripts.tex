% Librerías usadas
% Para poner pas cosas en español
\usepackage[spanish, es-tabla]{babel}
% Para usar imágenes
\usepackage{graphicx}
% Para tablas grandes
\usepackage{longtable}
% Para usar fuente helvética
\usepackage[scaled]{helvet}
\usepackage[T1]{fontenc}
\usepackage{textcomp}
\renewcommand\familydefault{\sfdefault}
% Para evitar saltos de página en cada capítulo insertado
\usepackage{etoolbox}
\makeatletter
\patchcmd{\chapter}{\if@openright\cleardoublepage\else\clearpage\fi}{}{}{}
\makeatother
%\usepackage{paralist}
\usepackage{circledsteps}
\usepackage{array}

\newcommand{\mychronodatestyle}[1]{%
\pgfmathparse{equal (sign(#1),-1)? int(abs( #1)):#1 }%
\pgfmathresult%
\pgfmathparse{equal(sign(#1),-1)? "BC":}%
 ~\pgfmathresult%
 }

\newcommand{\mychronovdatestyle}[1]{%
\pgfmathparse{equal(sign(#1),-1)? int(abs(#1)):#1 }%
\edef\tmp{\pgfmathresult}%
\pgfmathparse{equal(sign(#1),-1)? "BC":}%
 \kern-1.5pt\rotatebox[origin=left]{90}{--- \tmp~\pgfmathresult}%
 }

 \catcode`\!=11
 \def\eventyear{\!chreventyear}
 \catcode`\!=12

 \newcommand{\myeventdatestyle}[1]{%
 \pgfmathparse{equal(sign(\eventyear),-1)? int(abs(\eventyear)):"#1"}%
\pgfmathresult%
 \pgfmathparse{equal(sign(\eventyear),-1)? "BC":}%
 ~\pgfmathresult%
 }

% Paquete para armado de líneas temporales

% \usepackage{chronology}


% Genera formato apa
\usepackage[utf8]{inputenc}
\usepackage{csquotes}
\usepackage[backend=biber, style=apa]{biblatex}
\DefineBibliographyStrings{spanish}{ andothers = {et\addabbrvspace al\adddot}, and = {y}, }
% Añade archivo de bibliografía
\addbibresource{biblio.bib}
% Genera opción de usar markdown en latex
% \usepackage[pipeTables,tableCaptions,hybrid,smartEllipses,hashEnumerators]{markdown}
% Paquete float para evitar que las cosas se peguen donde se les dé su chingada gana
\usepackage{float}
\makeatletter
\renewcommand*{\fps@table}{H}
\makeatother
\floatstyle{plaintop}
\restylefloat{table}
\usepackage[justification=centering]{caption}
% Ruta de imágenes
\graphicspath{{imagenes/}}
\usepackage{setspace}
\doublespacing
% Ajuste de márgenes
\usepackage[left=2.5cm,top=2.5cm,right=2.5cm,bottom=2.5cm ]{geometry}
\usepackage{enumitem}
% Generación de scripts para todo
\usepackage{tikz}
\usetikzlibrary{shapes.geometric, arrows}

\setcounter{secnumdepth}{6}


% Macro para generar números en círculos
\newcommand{\cir}[1]{\tikz[baseline]{%
    \node[anchor=base, draw, circle, inner sep=0, minimum width=0.5em]{#1};}}

%librería tikz para generar diagramas
\tikzstyle{startstop} = [rectangle, rounded corners, minimum width=3cm, minimum height=1cm,text centered, text width=4.5cm, draw=black]
\tikzstyle{io} = [trapezium, trapezium left angle=70, trapezium right angle=110, minimum width=3cm, minimum height=1cm, text centered, draw=black]
\tikzstyle{process} = [rectangle, minimum width=3cm, minimum height=1cm, text centered, text width=3cm, draw=black]
\tikzstyle{decision} = [diamond, minimum width=3cm, minimum height=1cm, text centered, draw=black]
\tikzstyle{arrow} = [thick,->,>=stealth]
% Cambio en el título
\usepackage{titlesec, blindtext, color}
\definecolor{gray75}{gray}{0.75}
\newcommand{\hsp}{\hspace{20pt}}
% \titleformat{\chapter}[hang]{\huge\bfseries}{\thechapter{.}\hsp}{0pt}{\huge\bfseries}
\usepackage{amsmath}
\usepackage{amsfonts}
% \usepackage{amsrefs}
% \begin{figure}[]
%   \centering
%   \includegraphics[width=0.8\textwidth]{}
%   \caption{}
%   \label{fig:}
% \end{figure}
\usepackage{hyperref}

