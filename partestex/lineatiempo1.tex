\vspace{-2.54cm}
\begin{figure}[H]
    \scriptsize
    \chronoperiodecoloralternation{red,blue,cyan,green}
    \startchronology[startyear=1777,stopdate=false]
    \setupchronoevent{textstyle=\it}
    \chronoperiode[startdate=false,ifcolorbox=true,colorbox=white]{1777}{1945}{Era pre-computacional}
    \chronoperiode[startdate=false,ifcolorbox=true,colorbox=white]{1945}{1970}{}
    \chronoperiode[startdate=false,ifcolorbox=true,colorbox=white]{1970}{1981}{}
    \chronoperiode[startdate=false,ifcolorbox=true,colorbox=white]{1981}{2022}{Era actual}
    \chronoevent[textwidth=1.5cm,markdepth=30pt]{1777}{ \scriptsize Experimento de la aguja}
    \chronoevent[textwidth=1.3cm,markdepth=60pt]{1963}{ \scriptsize Creación de SIMSCRIPT}
    \chronoevent[textwidth=1.5cm,markdepth=20pt]{1940}{ \scriptsize Creación de ENIAC}
    \chronoevent[textwidth=1.5cm,markdepth=20pt]{1982}{ \scriptsize Creación de Automod}
    \chronoevent[textwidth=1.5cm,markdepth=-70pt]{1984}{ \scriptsize Desarrollo de Simple++}
    \chronoevent[textwidth=1.3cm,markdepth=60pt]{1994}{ \scriptsize Creación de Java}
    \chronoevent[textwidth=1.5cm,markdepth=-110pt]{1972}{ \scriptsize Creación de C}
    \chronoevent[textwidth=1.3cm,markdepth=-25pt]{1995}{ \scriptsize Creación de Python}
    \chronoevent[textwidth=1.7cm,markdepth=-40pt]{1945}{ \scriptsize Desarrollo de computadoras analógicas}
    \chronoevent[markdepth=-23pt]{1870}{\footnotesize Método de Laplace}
    \chronoevent[markdepth=25pt]{1855}{\footnotesize Prueba t-student}
\stopchronology
    \caption{Línea del tiempo de la historia de la simulación. \\ Tomado de \textcite{goldsman1} y \textcite{roberts1}}
    \label{fig:litiem1}
\end{figure}

